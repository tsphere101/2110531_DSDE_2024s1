\chapter{บทนำ}
\section{การกำหนดปัญหา}
ในโลกของการวิจัยทางวิชาการ
การแบ่งประเภทและการจำแนกวรรณกรรมมีบทบาทสำคัญในงานวรรณกรรม
เพื่อผู้อ่านสามารถค้นหาผลงานทางวิชาการที่เกี่ยวข้องได้นั้น 
จำเป็นต้องมีวิธีการแบ่งประเภทวรรณกรรมอย่างถูกต้องและแม่นยำ
ด้วยฐานข้อมูลอย่าง Scopus ที่มีบทความนับล้านบทความ
การแยกประเภทด้วยการการตีความโดยใช้มนุษย์จึงเป็นเรื่องที่ท้าทายและต้องใช้แรงงานอย่างมาก
วิธีแก้ปัญหาที่มีแนวโน้มที่ดีอย่างหนึ่งคือการจำแนกข้อความแบบหลายป้ายกำกับ หรือ Multi-label text classification

ปัญหา Multi-label text classification คือการจำแนกหมวดหมู่ของข้อความ โดยที่ข้อความหนึ่งสามารถเป็นไปได้หลากหลายหมวดหมู่ ปัญหาที่ได้รับในโจทย์จะเป็นการจำแนกบทความทางวิชาการในด้านวิศวกรรมศาสตร์ โดยแบ่งเป็น 18 สาขา ได้แก่ civil, environmental, biomedical, petroleum, metallurgical, mechanical, electrical, computer, optical, nano, chemical, materials, agricultural, education, industrial, safety, "mathematics and statistics", and material science.

\section{วัตถุประสงค์}
วัตถุประสงค์ของการทดลองในครั้งนี้คือการพัฒนาโมเดลการจำแนกข้อความหลายป้ายกำกับที่มีประสิทธิภาพซึ่งสามารถคาดการณ์หัวข้อต่าง ๆ ได้อย่างแม่นยำจากเอกสารบทคัดย่อและชื่องานวิจัย

เนื่องจากเอกสารแต่ละฉบับสามารถอยู่ในสาขาวิศวกรรมศาสตร์ที่แตกต่างกัน 18 สาขาแต่ละเอกสารสามารถอยู่ได้มากกว่า 1 สาขา โมเดลจัดหมวดหมู่จึงจำเป็นต้องเรียนรู้รูปแบบที่ซับซ้อนจากข้อความและกำหนดป้ายกำกับที่เหมาะสมซึ่งสะท้อนถึงลักษณะสหวิทยาการของเนื้อหา ซึ่งจะทำให้สามารถจัดหมวดหมู่วรรณกรรมทางวิศวกรรมศาสตร์ได้อย่างแม่นยำ ซึ่งจะช่วยให้นักวิจัยจัดระเบียบและค้นหาผลงานวิจัยที่เกี่ยวข้องในโดเมนวิศวกรรมศาสตร์ที่หลากหลายได้อย่างมีประสิทธิภาพ